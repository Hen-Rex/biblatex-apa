\documentclass[paper=a4]{article}
\usepackage[american]{babel}
\usepackage{csquotes}
\usepackage{shortvrb}
\usepackage{ifthen}
\usepackage{color}
\usepackage[retainmissing]{MnSymbol}
\usepackage[top=2.5cm]{geometry}

% Unicode
\usepackage{fontspec}
\setmainfont[Ligatures=TeX]{TeXGyrePagella}
\setsansfont[Ligatures=TeX]{TexGyreHeros}
\setmonofont[Ligatures=NoCommon]{TeXGyreCursor}
\MakeAutoQuote{«}{»}
\maxdeadcycles=1000
% Biblatex
\usepackage[style=apa]{biblatex}

\MakeShortVerb{\|}

\newcommand\apa[2][]{\ifthenelse{\equal{#1}{}}%
                       {\textcolor{blue}{\texttt{(APA #2)}}}%
                       {\textcolor{blue}{\texttt{(APA #2 Example #1)}}}}

\setlength{\parskip}{3ex}

\makeatletter
% This solution to detokenize leaving a space after the command and
% arguments is due to Bruno Le Floch on T.SE
\long\def\apaexi#1{%
  \catcode64=11
  \begingroup
  % Ensure that every character is preserved by \lowercase.
  \count@\z@
  \loop\ifnum\count@<256
    \lccode\count@\z@
    \advance\count@\@ne
  \repeat
  % Except spaces, changed to ^^A
  \lccode32=\@ne
  \lowercase{%
    \endgroup
    \expandafter\test@\detokenize{#1}\relax%
    \catcode64=12}}
% Then map {^^A => space, space =>} onto the string.
\def\test@#1{%
  \ifx#1\relax\test@end\fi
  \ifnum`#1=\@ne\space\else#1\fi
  \test@}
\def\test@end\fi#1\test@{\fi}

\def\apaex#1{\hspace*{-4em}\parbox{\textwidth}{\texttt{\small\apaexi{#1}}}\\$\rcurvearrowse$ \textbf{#1}}
\def\apaexs#1{\hbox{\texttt{\footnotesize\apaexi{#1}}} \textbf{\small #1}}
\makeatother

% This just makes it easier to find a specific (APA 7.x) example in the
% typeset references section
\reversemarginpar
\renewbibmacro*{begentry}{\marginpar{\footnotesize\textcolor{blue}{\thefield{entrykey}}}%
                          \ifkeyword{meta}{\textsuperscript{*}}{}}

\usepackage[colorlinks=true]{hyperref}
\setcounter{secnumdepth}{0}
\setcounter{tocdepth}{3}

%%%%%%%%%%%%%%%%%%%%%%%%%%%% END PREAMBLE %%%%%%%%%%%%%%%%%%%%%%%%%

\addglobalbib{../bibtex/bib/biblatex-apa-test-references.bib}

\begin{document}
\section{APA style examples}
This file typesets just about all useful examples from \apa{8}--\apa{10}.
Also included are some clarifications from the APA blog
(\url{http://blog.apastyle.org/apastyle/}). Please refer to the
|biblatex-apa-test-references.bib| file for details on the references
entries. The |.bib| key for each entry in the References section is listed
for convenience in the left margin. The keys are not arbitrary and consist
of the APA section they are taken from, followed by a colon,
followed by the example number. This makes it easier to cross-reference the
typeset examples with the commented |.bib| file.

\tableofcontents
\newpage
\section{APA 8--Citations}
\begin{refsection}[../bibtex/bib/biblatex-apa-test-citations.bib]
\noindent Please see accompanying file |biblatex-apa-test-citations.bib|
for the bibliographic entries which these examples use.

\noindent\apa{8.6}\\
Secondary sources. The example in APA 7th is inconsistent with other
examples of similar format which use a semi-colon:\\
\apaex{\parencites{8.6a}[as cited in][]{8.6b}}\\\\
\noindent To obtain the exact format, if desired, wrap the citation in:\\
|\DeclareDelimFormat{multicitedelim}{\addcomma\space}|\\
|\DeclareDelimFormat{multicitedelim}{\addsemicolon\space}|\\
\DeclareDelimFormat{multicitedelim}{\addcomma\space}\\
\apaex{\parencites{8.6a}[as cited in][]{8.6b}}\\
\DeclareDelimFormat{multicitedelim}{\addsemicolon\space}\\
\noindent Primary source with no date:\\
\apaex{Allport's diary \parencite[as cited in][]{8.6c}}

\noindent\apa{8.9}\\
Such cases are just typed out--they have no Reference section entry and
don't therefore have a bibliography database entry.

\noindent\apa{8.10}\\
Simple cite. «Jr.» suffix is not shown (bib entry for this example has a suffix):\\
\apaex{\textcite{8.10a}}\\

\noindent Testing seasonal periodical citations--should be ignored and only
year printed:\\
\apaex{\textcite{8.10b}}

\noindent\apa{8.11}\\
Within a paragraph, not in the narrative sense:\\
\apaex{\autocite{8.11a}}

\noindent Within parentheses, use the |\nptextcite| command which is equivalent to
the |\textcite| command but omits the parenthesis and uses commas instead.
See the |biblatex-apa| docs.\\
\apaex{(\nptextcite[see][for more detail]{8.11a})}\\
\apaex{(e.g., falsely balanced news coverage; \nptextcite{8.11a})}\\
\apaex{\textcite{8.11a} noted the dangers of falsely balanced news coverage.}\\
\apaex{In \citeyear{8.11a}, \citeauthor{8.11a} noted the dangers of falsely balanced news coverage.}

\noindent\apa{8.12}\\
Citing multiple works:\\
\apaex{\parencite{8.12a,8.12b,8.12c}}\\
\apaex{\parencite{8.12d,8.12e,8.12f,8.12g}}\\
\apaex{\parencite{8.12h,8.12i,8.12j,8.12k}}\\
\apaex{\parencite{8.12l,8.12m,8.12n}}\\
\apaex{\parencites{8.12o}[see also][]{8.12p,8.12q, 8.12r}}

\noindent «in press» should have hyphen before disambiguating part of label\\
\apaex{\parencite{8.12s,8.12t,8.12u}}

\noindent\apa{8.13}\\
Classical and part works:\\
\apaex{\parencite[10]{8.13a}}\\
\apaex{\parencite[Chapter 3]{8.13b}}\\
\apaex{\parencite[3--17]{8.13c}}\\
\apaex{\parencite[paras. 2--3]{8.13d}}\\
\apaex{\parencite[Table 1]{8.13e}}\\
\apaex{\parencite[Slide 7]{8.13f}}\\
\apaex{\parencite[1:30:40]{8.13g}}\\
\apaex{\parencite[1 Cor. 13:1]{8.13h}}\\
\apaex{\parencite[Part IV]{8.13i}}\\
\apaex{\parencite[\nopp 1.3.36--37]{8.13j}}\\\\
Note the use of |\nopp| in the Shakespeare example to suppress the
pagination prefix since the part specification is auto-detected as a range
by |biblatex| which it then assumes is a page range. Note also the date
format for the Aristotle example, see the reference file for the data
format--no special formatting is required by the user as |biblatex| fully
supports the relevant parts of the ISO 8601 specification and can
parse/format such dates automatically.

\noindent\apa{8.14}\\
Use |SHORTTITLE| field of the entry if it exists:\\
\apaex{\parencite{8.14a}}\\\\
Articles use quotes instead of italics:\\
\apaex{\parencite{8.14b}}\\\\
Citing anonymous author:\\
\apaex{\parencite{8.14c}}

\noindent\apa{8.15}\\
Citations with reprint/reissue dates:\\
\apaex{\textcite{8.15a}}\\
\apaex{\parencite{8.15b}}

\noindent\apa{8.16}\\
This is not a strict rule and so simply use the lower-level citation
commands:\\
\apaex{\citeauthor{8.11a}}

\noindent\apa{8.17}\\
Basic in-text citation styles:\\
\begin{table}
\begin{tabular}{p{0.33\textwidth}p{0.33\textwidth}p{0.33\textwidth}}
  \textbf{Author type} & \textbf{Parenthetical citation} & \textbf{Narrative citation}\\\\
  \hline
  \\
  One author & \apaexs{\parencite{8.17a}} & \apaexs{\textcite{8.17a}}\\\\
  Two authors & \apaexs{\parencite{8.17b}} & \apaexs{\textcite{8.17b}}\\\\
  Three or more authors & \apaexs{\parencite{8.17c}} & \apaexs{\textcite{8.17c}}\\\\
  Group author with abbreviation & &\\
  ~~First citation & \apaexs{\parencite{8.17d}}\citeresetapa & \apaexs{\textcite{8.17d}}\\\\
  ~~Subsequent citations & \apaexs{\parencite{8.17d}} & \apaexs{\textcite{8.17d}}\\\\
  Group author without abbreviation & \apaexs{\parencite{8.17e}} & \apaexs{\textcite{8.17e}}\\\\
  \hline
\end{tabular}
\caption{APA Table 8.1}
\label{tab:8.1}
\end{table}

\noindent Authors in narrative citations are separated by «and».
However, in parenthetical cites, authors are separated by «\&»:\\
\apaex{\parencite{8.17f}}\\
\apaex{\textcite{8.17g}}\\
\apaex{\parencite{8.17h}}

\noindent Automatic list disambiguation for ambiguous truncations. Note
that «et al» is plural so it will only truncate two or more names. If it
would truncate just one name, we just give the name:\\
\apaex{\textcite{8.18a}}\\
\apaex{\textcite{8.18b}}\\
\apaex{\textcite{8.18c}}\\
\apaex{\textcite{8.18d}}

\noindent Automatic list disambiguation is only for ambiguous author lists
in the same year. This is implied in \apa{8.18} by the phrase «the same
in-text citation form». So these are not disambiguated:\\
\apaex{\textcite{8.18e}}\\
\apaex{\textcite{8.18f}}

\noindent\apa{8.19}\\
Citations of works by same authors in the same year:\\
\apaex{\parencite{8.19a}}\\
\apaex{\textcite{8.19b}}\\
\apaex{\parencite{8.19c,8.19d}}

\noindent\apa{8.20}\\
Citations of an entry with an author who shares a surname with another
entry always appears with initials when primary author:\\
\apaex{\parencite{8.20a,8.20b}}

\noindent When initials are also the same, revert to standard format:\\
\apaex{\parencite{8.20c,8.20d}}

\noindent Only the primary author should have initials:\\
\apaex{\parencite{8.20f,8.20g,8.20h,8.20i,8.20j}}\\

\noindent Authors in same reference with same surname:\\
\apaex{\parencite{8.20e}}

\noindent\apa{8.21}\\
Group author abbreviations should work between citations of different works by
the author:\\
\apaex{\parencite{8.21a}}\\
\apaex{\parencite{8.21a}}\\
\apaex{\parencite{8.21b}}

\noindent Testing suffices\\
\apaex{\textcite{stest1}}\\
\apaex{\textcite{stest2}}

\noindent Testing prefices\\
\apaex{\textcite{pretest}}\\
\apaex{\textcite{nopretest}}

% presufftest2 and nopresufftest are the same name and so even though
% useprefix=false nopresufftest, hashes are the same so there is an extradate
\noindent Testing prefices plus suffices\\
\apaex{\textcite{presufftest}}\\
\apaex{\textcite{presufftest2}}\\
\apaex{\textcite{nopresufftest}}

\noindent Testing «and others»\\
\apaex{\textcite{ao1}}\\
\apaex{\textcite{ao2}}

% Testing name elision in cites
% That is, \parencite{kingetal2005a,weissetal2007a,weissetal2009a}
% Should give:
% (King, Weiss, & Farmer, 2005; Weiss, King, & Hopkins, 2007; Weiss et al., 2009)
% and not
% (King, Weiss, & Farmer, 2005; Weiss, King, & Hopkins, 2007, 2009)
% which happens if elision is based on namehash because the mincitenames
% truncation of the 2007 and 2009 names is the same
\noindent{Testing name elision in cites}\\
\apaex{\parencite{kingetal2005a,weissetal2007a,weissetal2009a}}

\noindent Testing citation formats for complete dates\\
\apaex{\textcite{Ludwig2014}}

\noindent Testing month ranges with same months, different days\\
\apaex{\textcite{daterange1}}

\noindent Testing month ranges with different months\\
\apaex{\textcite{daterange2}}

\noindent Testing fullcite\\
\apaex{\fullcite{FC1}}\\
\apaex{\fullcitebib{FC1}}

\printbibliography[title={APA 8 Citation References}]

\end{refsection}
\newpage
\section{APA 9--General Reference List}

% (APA 9.8)
\begin{refsection}

\noindent\apa{9.8}\\
\apaex{\parencite{9.8:1}}\\
\apaex{\parencite{9.8:2}}\\
\apaex{\parencite{9.8:3}}\\
\apaex{\parencite{9.8:4}}\\
\apaex{\parencite{9.8:5}}\\
\apaex{\parencite{9.8:6}}\\
\apaex{\parencite{9.8:7}}\\
\apaex{\parencite{9.8:8}}\\
\apaex{\parencite{9.8:9}}\\
\apaex{\parencite{9.8:10}}\\
\apaex{\parencite{9.8:11}}

\apaex{\textcite{9.8:1}}\\
\apaex{\textcite{9.8:2}}\\
\apaex{\textcite{9.8:3}}\\
\apaex{\textcite{9.8:4}}\\
\apaex{\textcite{9.8:5}}\\
\apaex{\textcite{9.8:6}}\\
\apaex{\textcite{9.8:7}}\\
\apaex{\textcite{9.8:8}}\\
\apaex{\textcite{9.8:9}}\\
\apaex{\textcite{9.8:10}}\\
\apaex{\textcite{9.8:11}}

\printbibliography[title={APA 9.8--Format of the Author Element}]

\noindent\textbf{Note:} The username feature is nicely managed by |biber|'s
annotation feature which allows arbitrary strings to be attached to any
list member.

\end{refsection}
\newpage

% (APA 9.14)
\begin{refsection}
  
\nocite{9.14:1,9.14:2}
    
\printbibliography[title={APA 9.14--Misc Date Elements}]

\end{refsection}
\newpage

% (APA 9.44)
\begin{refsection}
  
\nocite{9.44:1a,9.44:1b,9.44:2a,9.44:2b,9.44:2c,9.44:3a,9.44:3b,9.44:4a,9.44:4b,%
        9.44:5a,9.44:5b,9.44:6a,9.44:6b,9.44:7a,9.44:7b,9.44:7c,9.44:8a,9.44:8b,9.44:8c,%
        9.44:9a,9.44:9b,9.44:10a,9.44:10b,9.44:10c,9.44:11a,9.44:11b,9.44:12a,9.44:12b}
    
\printbibliography[title={APA 9.44--Order of Works in a Reference List}]

\noindent\textbf{Note:} The rather silly requirement for the «Santiago» examples are
done via |SORTNAME|.

\end{refsection}
\newpage
% (APA 9.46)
\begin{refsection}
  
\nocite{9.46:1,9.46:2,9.46:3,9.46:4,9.46:5,9.46:6,9.46:7,9.46:8,9.46:9,9.46:10,%
        9.46:11,9.46:12,9.46:13}

\printbibliography[title={APA 9.46--Order of Multiple Works by the Same First Author}]

\end{refsection}
\newpage
% (APA 9.47)
\begin{refsection}

\nocite{9.47:1,9.47:2,9.47:3,9.47:4,9.47:5,9.47:6,9.47:7,9.47:8,9.47:9,9.47:10,%
        9.47:11,9.47:12,9.47:13}

\printbibliography[title={APA 9.47--Order of Works With the Same Author and
Same Date}]

\noindent\textbf{Note:} Examples added for no dates, «in press» requirements and
also ignoring of «The», «An» and «A» in titles for sorting purposes.

\end{refsection}
\newpage
% (APA 9.48)
\begin{refsection}

\nocite{9.48:1,9.48:2}

\printbibliography[title={APA 9.48--Order of Works by First Authors With
  the Same Surname}]

\end{refsection}
\newpage
% (APA 9.49)
\begin{refsection}

\nocite{9.49:1,9.49:2,9.49:3,9.49:4,9.49:5,9.49:6}

\printbibliography[title={APA 9.49--Order of Works with No Author or an
  Anonymous Author}]

\noindent\textbf{Note:} Examples added for anonymous authors and numeral
expansion/series sorting using |SORTTITLE|.

\end{refsection}
\newpage
% (APA 9.51)
\begin{refsection}

\nocite{9.51:1,9.51:2}

\printbibliography[title={APA 9.51--Annotated Bibliographies}]

\end{refsection}
\newpage
% (APA 9.52)
\begin{refsection}

\nocite{9.52:1,9.52:2,9.52:3,9.52:4}

\printbibliography[title={APA 9.52--References Included in a Meta-Analysis}]

\noindent\textbf{Note:} Use the |meta| keyword in the |KEYWORDS| field to
indicate works included in a Meta-Analysis.

\end{refsection}
\newpage
\section{APA 10--References Examples}
\subsection{APA 10.1--Periodicals}

\begin{refsection}

\noindent\apa{10.1 Example 1}\\
\apaex{\parencite{10.1:1}}\\
\apaex{\textcite{10.1:1}}

\noindent\apa{10.1 Example 2}\\
\apaex{\parencite{10.1:2}}\\
\apaex{\textcite{10.1:2}}

\noindent\apa{10.1 Example 3}\\
\apaex{\parencite{10.1:3a,10.1:3b}}\\
\apaex{\textcite{10.1:3a} and \textcite{10.1:3b}}

\noindent\apa{10.1 Example 4}\\
\apaex{\parencite{10.1:4}}\\
\apaex{\textcite{10.1:4}}

\noindent\apa{10.1 Example 5}\\
\apaex{\parencite{10.1:5}}\\
\apaex{\textcite{10.1:5}}

\noindent\apa{10.1 Example 6}\\
\apaex{\parencite{10.1:6}}\\
\apaex{\textcite{10.1:6}}

\noindent\apa{10.1 Example 7}\\
\apaex{\parencite{10.1:7}}\\
\apaex{\textcite{10.1:7}}

\noindent\apa{10.1 Example 8}\\
\apaex{\parencite{10.1:8}}\\
\apaex{\textcite{10.1:8}}

\noindent\apa{10.1 Example 9}\\
\apaex{\parencite{10.1:9}}\\
\apaex{\textcite{10.1:9}}

\noindent\apa{10.1 Example 10}\\
\apaex{\parencite{10.1:10}}\\
\apaex{\textcite{10.1:10}}

\noindent\apa{10.1 Example 11}\\
\apaex{\parencite{10.1:11}}\\
\apaex{\textcite{10.1:11}}

\noindent\apa{10.1 Example 12}\\
\apaex{\parencite{10.1:12a,10.1:12b}}\\
\apaex{\textcite{10.1:12a} and \textcite{10.1:12b}}

\noindent\apa{10.1 Example 13}\\
\apaex{\parencite{10.1:13}}\\
\apaex{\textcite{10.1:13}}

\noindent\apa{10.1 Example 14}\\
\apaex{\parencite{10.1:14}}\\
\apaex{\textcite{10.1:14}}

\noindent\apa{10.1 Example 15}\\
\apaex{\parencite{10.1:15a,10.1:15b,10.1:15c}}\\
\apaex{\textcite{10.1:15a}, \textcite{10.1:15b}, and \textcite{10.1:15c}}

\noindent\apa{10.1 Example 16}\\
\apaex{\parencite{10.1:16a,10.1:16b}}\\
\apaex{\textcite{10.1:16a} and \textcite{10.1:16b}}

\noindent\apa{10.1 Example 17}\\
\apaex{\parencite{10.1:17}}\\
\apaex{\textcite{10.1:17}}

\noindent\apa{10.1 Example 18}\\
\apaex{\parencite{10.1:18}}\\
\apaex{\textcite{10.1:18}}

\noindent\apa{10.1 Example 19}\\
\apaex{\parencite{10.1:19}}\\
\apaex{\textcite{10.1:19}}

\printbibliography[title={APA 10.1 References}]

\end{refsection}
\newpage
\subsection{APA 10.2--Books and Reference Works}

\begin{refsection}

\noindent\apa{10.2 Example 20}\\
\apaex{\parencite{10.2:20}}\\
\apaex{\textcite{10.2:20}}

\noindent\apa{10.2 Example 21}\\
\apaex{\parencite{10.2:21}}\\
\apaex{\textcite{10.2:21}}

\noindent\apa{10.2 Example 22}\\
\apaex{\parencite{10.2:22a,10.2:22b}}\\
\apaex{\textcite{10.2:22a} and \textcite{10.2:22b}}

\noindent\apa{10.2 Example 23}\\
\apaex{\parencite{10.2:23}}\\
\apaex{\textcite{10.2:23}}

\noindent\apa{10.2 Example 24}\\
\apaex{\parencite{10.2:24}}\\
\apaex{\textcite{10.2:24}}

\noindent\apa{10.2 Example 25}\\
\apaex{\parencite{10.2:25}}\\
\apaex{\textcite{10.2:25}}

\noindent\apa{10.2 Example 26}\\
\apaex{\parencite{10.2:26}}\\
\apaex{\textcite{10.2:26}}

\noindent\apa{10.2 Example 27}\\
\apaex{\parencite{10.2:27a,10.2:27b}}\\
\apaex{\textcite{10.2:27a} and \textcite{10.2:27b}}

\noindent\apa{10.2 Example 28}\\
\apaex{\parencite{10.2:28}}\\
\apaex{\textcite{10.2:28}}

\noindent\apa{10.2 Example 29}\\
\apaex{\parencite{10.2:29a,10.2:29b}}\\
\apaex{\textcite{10.2:29a} and \textcite{10.2:29b}}

\noindent\apa{10.2 Example 30}\\
\apaex{\parencite{10.2:30a,10.2:30b}}\\
\apaex{\textcite{10.2:30a} and \textcite{10.2:30b}}

\noindent\apa{10.2 Example 31}\\
\apaex{\parencite{10.2:31}}\\
\apaex{\textcite{10.2:31}}

\noindent\apa{10.2 Example 32}\\
The examples for citing such manuals are a mess in the 7th edition with a
language-dependent possessive in the narrative examples. However, these
rules are not strict according to the comments and can be constructed with
the help of lower-level commands:\\\\
\apaex{\citetitle{10.2:32a} (\citefield{10.2:32a}{edition};
  \citefield{10.2:32a}{shorthand}; \citeauthor{10.2:32a}, \citeyear{10.2:32a})}\\
\apaex{\citetitle{10.2:32b} (\citefield{10.2:32b}{edition};
  \citefield{10.2:32b}{shorthand}; \citeauthor{10.2:32b}, \citeyear{10.2:32b})}\\\\
\apaex{\citeauthor{10.2:32a}'s (\citeyear{10.2:32a}) \citetitle{10.2:32a}
  (\citefield{10.2:32a}{edition}; \citefield{10.2:32a}{shorthand})}\\
\apaex{\citeauthor{10.2:32b}'s (\citeyear{10.2:32b}) \citetitle{10.2:32b}
  (\citefield{10.2:32b}{edition}; \citefield{10.2:32b}{shorthand})}\\\\

\noindent Then for subsequent cites, use the normal commands:\\
\apaex{\parencite{10.2:32a,10.2:32b}}\\
\apaex{\textcite{10.2:32a} and \textcite{10.2:32b}}

\noindent\apa{10.2 Example 33}\\
\apaex{\parencite{10.2:33a,10.2:33b,10.2:33c}}\\
\apaex{\textcite{10.2:33a} and \textcite{10.2:33b} and \textcite{10.2:33c}}

\noindent\apa{10.2 Example 34}\\
\apaex{\parencite{10.2:34}}\\
\apaex{\textcite{10.2:34}}

\noindent\apa{10.2 Example 35}\\
\apaex{\parencite{10.2:35a,10.2:35b,10.2:35c}}\\
\apaex{\textcite{10.2:35a} and \textcite{10.2:35b} and \textcite{10.2:35c}}

\noindent\apa{10.2 Example 36}\\
\apaex{\parencite{10.2:36}}\\
\apaex{\textcite{10.2:36}}

\noindent\apa{10.2 Example 37}\\
\apaex{\parencite{10.2:37}}\\
\apaex{\textcite{10.2:37}}

\printbibliography[title={APA 10.2 References}]

\end{refsection}
\newpage
\subsection{APA 10.3--Edited Book Chapters and Entries in Reference Works}

\begin{refsection}

\noindent\apa{10.3 Example 38}\\
\apaex{\parencite{10.3:38}}\\
\apaex{\textcite{10.3:38}}

\noindent\apa{10.3 Example 39}\\
\apaex{\parencite{10.3:39}}\\
\apaex{\textcite{10.3:39}}

\noindent\apa{10.3 Example 40}\\
\apaex{\parencite{10.3:40}}\\
\apaex{\textcite{10.3:40}}

\noindent\apa{10.3 Example 41}\\
\apaex{\parencite{10.3:41}}\\
\apaex{\textcite{10.3:41}}

\noindent\apa{10.3 Example 42}\\
\apaex{\parencite{10.3:42}}\\
\apaex{\textcite{10.3:42}}

\noindent\apa{10.3 Example 43}\\
\apaex{\parencite{10.3:43}}\\
\apaex{\textcite{10.3:43}}

\noindent\apa{10.3 Example 44}\\
\apaex{\parencite{10.3:44}}\\
\apaex{\textcite{10.3:44}}

\noindent\apa{10.3 Example 45}\\
\apaex{\parencite{10.3:45}}\\
\apaex{\textcite{10.3:45}}

\noindent\apa{10.3 Example 46}\\
\apaex{\parencite{10.3:46}}\\
\apaex{\textcite{10.3:46}}

\noindent\apa{10.3 Example 47}\\
\apaex{\parencite{10.3:47a,10.3:47b}}\\
\apaex{\textcite{10.3:47a} and \textcite{10.3:47b}}

\noindent\apa{10.3 Example 48}\\
\apaex{\parencite{10.3:48}}\\
\apaex{\textcite{10.3:48}}

\noindent\apa{10.3 Example 49}\\
\apaex{\parencite{10.3:49}}\\
\apaex{\textcite{10.3:49}}

\printbibliography[title={APA 10.3 References}]

\end{refsection}
\newpage
\subsection{APA 10.4--Reports and Gray Literature}

\begin{refsection}

\noindent\apa{10.4 Example 50}\\
\apaex{\parencite{10.4:50a,10.4:50b,10.4:50c}}\\
\apaex{\textcite{10.4:50a} and \textcite{10.4:50b} and \textcite{10.4:50c}}

\noindent\apa{10.4 Example 51}\\
\apaex{\parencite{10.4:51a,10.4:51b}}\\
\apaex{\textcite{10.4:51a} and \textcite{10.4:51b}}

\noindent\apa{10.4 Example 52}\\
\apaex{\parencite{10.4:52}}\\
\apaex{\textcite{10.4:52}}

\noindent\apa{10.4 Example 53}\\
\apaex{\parencite{10.4:53}}\\
\apaex{\textcite{10.4:53}}

\noindent\apa{10.4 Example 54}\\
\apaex{\parencite{10.4:54}}\\
\apaex{\textcite{10.4:54}}

\noindent\apa{10.4 Example 55}\\
\apaex{\parencite{10.4:55a,10.4:55b,10.4:55c}}\\
\apaex{\textcite{10.4:55a} and \textcite{10.4:55b} and \textcite{10.4:55c}}

\noindent\apa{10.4 Example 56}\\
\apaex{\parencite{10.4:56}}\\
\apaex{\textcite{10.4:56}}

\noindent\apa{10.4 Example 57}\\
\apaex{\parencite{10.4:57}}\\
\apaex{\textcite{10.4:57}}

\noindent\apa{10.4 Example 58}\\
\apaex{\parencite{10.4:58}}\\
\apaex{\textcite{10.4:58}}

\noindent\apa{10.4 Example 59}\\
\apaex{\parencite{10.4:59}}\\
\apaex{\textcite{10.4:59}}

\printbibliography[title={APA 10.4 References}]
\end{refsection}
\newpage
\subsection{APA 10.5--Conference Sessions and Presentations}

\begin{refsection}

\noindent\apa{10.5 Example 60}\\
\apaex{\parencite{10.5:60}}\\
\apaex{\textcite{10.5:60}}

\noindent\apa{10.5 Example 61}\\
\apaex{\parencite{10.5:61}}\\
\apaex{\textcite{10.5:61}}

\noindent\apa{10.5 Example 62}\\
\apaex{\parencite{10.5:62}}\\
\apaex{\textcite{10.5:62}}

\noindent\apa{10.5 Example 63}\\
\apaex{\parencite{10.5:63}}\\
\apaex{\textcite{10.5:63}}

\noindent The following are proper proceedings types and are included in a web
addendum here: \url{https://apastyle.apa.org/style-grammar-guidelines/references/examples/conference-proceeding-references}

\noindent\apa{10.5 Addendum Example 1}\\
\apaex{\parencite{10.5:A1}}\\
\apaex{\textcite{10.5:A1}}

\noindent\apa{10.5 Addendum Example 2}\\
\apaex{\parencite{10.5:A2}}\\
\apaex{\textcite{10.5:A2}}

\noindent\apa{10.5 Addendum Example 3}\\
\apaex{\parencite{10.5:A3}}\\
\apaex{\textcite{10.5:A3}}

\printbibliography[title={APA 10.5 References}]
\end{refsection}
\newpage
\subsection{APA 10.6--Dissertations and Theses}

\begin{refsection}

\noindent\apa{10.6 Example 64}\\
\apaex{\parencite{10.6:64}}\\
\apaex{\textcite{10.6:64}}

\noindent\apa{10.6 Example 65}\\
\apaex{\parencite{10.6:65}}\\
\apaex{\textcite{10.6:65}}

\noindent\apa{10.6 Example 66}\\
\apaex{\parencite{10.6:66}}\\
\apaex{\textcite{10.6:66}}

\printbibliography[title={APA 10.6 References}]
\end{refsection}
\newpage
\subsection{APA 10.7--Reviews}

\begin{refsection}

\noindent\apa{10.7 Example 67}\\
\apaex{\parencite{10.7:67}}\\
\apaex{\textcite{10.7:67}}

\noindent\apa{10.7 Example 68}\\
\apaex{\parencite{10.7:68}}\\
\apaex{\textcite{10.7:68}}

\noindent\apa{10.7 Example 69}\\
\apaex{\parencite{10.7:69}}\\
\apaex{\textcite{10.7:69}}

\printbibliography[title={APA 10.7 References}]
\end{refsection}
\newpage
\subsection{APA 10.8--Unpublished Works and Informally Published Works}

\begin{refsection}

\noindent\apa{10.8 Example 70}\\
\apaex{\parencite{10.8:70}}\\
\apaex{\textcite{10.8:70}}

\noindent\apa{10.8 Example 71}\\
\apaex{\parencite{10.8:71}}\\
\apaex{\textcite{10.8:71}}

\noindent\apa{10.8 Example 72}\\
\apaex{\parencite{10.8:72}}\\
\apaex{\textcite{10.8:72}}

\noindent\apa{10.8 Example 73}\\
\apaex{\parencite{10.8:73a,10.8:73b}}\\
\apaex{\textcite{10.8:73a} and \textcite{10.8:73b}}

\noindent\apa{10.8 Example 74}\\
\apaex{\parencite{10.8:74}}\\
\apaex{\textcite{10.8:74}}

\printbibliography[title={APA 10.8 References}]
\end{refsection}
\newpage
\subsection{APA 10.9--Data Sets}

\begin{refsection}

\noindent\apa{10.9 Example 75}\\
\apaex{\parencite{10.9:75a,10.9:75b,10.9:75c}}\\
\apaex{\textcite{10.9:75a} and \textcite{10.9:75b} and \textcite{10.9:75c}}

\noindent\apa{10.9 Example 76}\\
\apaex{\parencite{10.9:76a,10.9:76b}}\\
\apaex{\textcite{10.9:76a} and \textcite{10.9:76b}}

\printbibliography[title={APA 10.9 References}]
\end{refsection}
\newpage
\subsection{APA 10.10--Computer Software, Mobile Apps, Apparatuses, and Equipment}

\begin{refsection}

\noindent\apa{10.10 Example 77}\\
\apaex{\parencite{10.10:77}}\\
\apaex{\textcite{10.10:77}}

\noindent\apa{10.10 Example 78}\\
\apaex{\parencite{10.10:78a,10.10:78b}}\\
\apaex{\textcite{10.10:78a} and \textcite{10.10:78b}}

\noindent\apa{10.10 Example 79}\\
\apaex{\parencite{10.10:79}}\\
\apaex{\textcite{10.10:79}}

\noindent\apa{10.10 Example 80}\\
\apaex{\parencite{10.10:80}}\\
\apaex{\textcite{10.10:80}}

\printbibliography[title={APA 10.10 References}]
\end{refsection}
\newpage
\subsection{APA 10.11--Tests, Scales, and Inventories}

\begin{refsection}

\noindent\apa{10.11 Example 81}\\
\apaex{\parencite{10.11:81}}\\
\apaex{\textcite{10.11:81}}

\noindent\apa{10.11 Example 82}\\
\apaex{\parencite{10.11:82}}\\
\apaex{\textcite{10.11:82}}

\noindent\apa{10.11 Example 83}\\
\apaex{\parencite{10.11:83a,10.11:83b}}\\
\apaex{\textcite{10.11:83a} and \textcite{10.11:83b}}

\printbibliography[title={APA 10.11 References}]
\end{refsection}
\newpage
\subsection{APA 10.12--Audiovisual Works}

\begin{refsection}

\noindent\apa{10.12 Example 84}\\
\apaex{\parencite{10.12:84a,10.12:84b,10.12:84c}}\\
\apaex{\textcite{10.12:84a} and \textcite{10.12:84b} and \textcite{10.12:84c}}

\noindent\apa{10.12 Example 85}\\
\apaex{\parencite{10.12:85}}\\
\apaex{\textcite{10.12:85}}

\noindent\apa{10.12 Example 86}\\
\apaex{\parencite{10.12:86}}\\
\apaex{\textcite{10.12:86}}

\noindent\apa{10.12 Example 87}\\
\apaex{\parencite{10.12:87a,10.12:87b}}\\
\apaex{\textcite{10.12:87a} and \textcite{10.12:87b}}

\noindent\apa{10.12 Example 88}\\
\apaex{\parencite{10.12:88a,10.12:88b}}\\
\apaex{\textcite{10.12:88a} and \textcite{10.12:88b}}

\noindent\apa{10.12 Example 89}\\
\apaex{\parencite{10.12:89}}\\
\apaex{\textcite{10.12:89}}

\noindent\apa{10.12 Example 90}\\
\apaex{\parencite{10.12:90a,10.12:90b,10.12:90c}}\\
\apaex{\textcite{10.12:90a} and \textcite{10.12:90b} and \textcite{10.12:90c}}

\printbibliography[title={APA 10.12 References}]
\end{refsection}
\newpage
\subsection{APA 10.13--Audio Works}

\begin{refsection}

\noindent\apa{10.13 Example 91}\\
\apaex{\parencite{10.13:91a,10.13:91b}}\\
\apaex{\textcite{10.13:91a} and \textcite{10.13:91b}}

\noindent\apa{10.13 Example 92}\\
\apaex{\parencite{10.13:92a,10.13:92b,10.13:92c,10.13:92d}}\\
\apaex{\textcite{10.13:92a} and \textcite{10.13:92b} and
  \textcite{10.13:92c} and \textcite{10.13:92d}}

\noindent\apa{10.13 Example 93}\\
\apaex{\parencite{10.13:93}}\\
\apaex{\textcite{10.13:93}}

\noindent\apa{10.13 Example 94}\\
\apaex{\parencite{10.13:94}}\\
\apaex{\textcite{10.13:94}}

\noindent\apa{10.13 Example 95}\\
\apaex{\parencite{10.13:95}}\\
\apaex{\textcite{10.13:95}}

\noindent\apa{10.13 Example 96}\\
\apaex{\parencite{10.13:96}}\\
\apaex{\textcite{10.13:96}}

\printbibliography[title={APA 10.13 References}]
\end{refsection}
\newpage
\subsection{APA 10.14--Visual Works}

\begin{refsection}

\noindent\apa{10.14 Example 97}\\
\apaex{\parencite{10.14:97a,10.14:97b}}\\
\apaex{\textcite{10.14:97a} and \textcite{10.14:97b}}

\noindent\apa{10.14 Example 98}\\
\apaex{\parencite{10.14:98}}\\
\apaex{\textcite{10.14:98}}

\noindent\apa{10.14 Example 99}\\
\apaex{\parencite{10.14:99}}\\
\apaex{\textcite{10.14:99}}

\noindent\apa{10.14 Example 100}\\
\apaex{\parencite{10.14:100a,10.14:100b}}\\
\apaex{\textcite{10.14:100a} and \textcite{10.14:100b}}

\noindent\apa{10.14 Example 101}\\
\apaex{\parencite{10.14:101a,10.14:101b}}\\
\apaex{\textcite{10.14:101a} and \textcite{10.14:101b}}

\noindent\apa{10.14 Example 102}\\
\apaex{\parencite{10.14:102a,10.14:102b,10.14:102c}}\\
\apaex{\textcite{10.14:102a} and \textcite{10.14:102b} and \textcite{10.14:102c}}

\printbibliography[title={APA 10.14 References}]
\end{refsection}
\newpage
\subsection{APA 10.15--Social Media}

\begin{refsection}

\noindent\apa{10.15 Example 103}\\
\apaex{\parencite{10.15:103a,10.15:103b,10.15:103c}}\\
\apaex{\textcite{10.15:103a} and \textcite{10.15:103b} and \textcite{10.15:103c}}

\noindent\apa{10.15 Example 104}\\
\apaex{\parencite{10.15:104}}\\
\apaex{\textcite{10.15:104}}

\noindent\apa{10.15 Example 105}\\
\apaex{\parencite{10.15:105a,10.15:105b,10.15:105c}}\\
\apaex{\textcite{10.15:105a} and \textcite{10.15:105b} and \textcite{10.15:105c}}

\noindent\apa{10.15 Example 106}\\
\apaex{\parencite{10.15:106}}\\
\apaex{\textcite{10.15:106}}

\noindent\apa{10.15 Example 107}\\
\apaex{\parencite{10.15:107}}\\
\apaex{\textcite{10.15:107}}

\noindent\apa{10.15 Example 108}\\
\apaex{\parencite{10.15:108}}\\
\apaex{\textcite{10.15:108}}

\noindent\apa{10.15 Example 109}\\
\apaex{\parencite{10.15:109}}\\
\apaex{\textcite{10.15:109}}

\printbibliography[title={APA 10.15 References}]
\end{refsection}
\newpage
\subsection{APA 10.16--Webpages and Websites}

\begin{refsection}

\noindent\apa{10.16 Example 110}\\
\apaex{\parencite{10.16:110a,10.16:110b}}\\
\apaex{\textcite{10.16:110a} and \textcite{10.16:110b}}

\noindent\apa{10.16 Example 111}\\
\apaex{\parencite{10.16:111a,10.16:111b}}\\
\apaex{\textcite{10.16:111a} and \textcite{10.16:111b}}

\noindent\apa{10.16 Example 112}\\
\apaex{\parencite{10.16:112}}\\
\apaex{\textcite{10.16:112}}

\noindent\apa{10.16 Example 113}\\
\apaex{\parencite{10.16:113a,10.16:113b}}\\
\apaex{\textcite{10.16:113a} and \textcite{10.16:113b}}

\noindent\apa{10.16 Example 114}\\
\apaex{\parencite{10.16:114}}\\
\apaex{\textcite{10.16:114}}

\printbibliography[title={APA 10.16 References}]
\end{refsection}
\newpage
\subsection{APA 11.4--Cases or Court Decisions}

\begin{refsection}

\noindent\apa{11.4 Example 1}\\
\apaex{\parencite{11.4:1}}\\
\apaex{\textcite{11.4:1}}

\noindent\apa{11.4 Example 2}\\
\apaex{\parencite{11.4:2}}\\
\apaex{\textcite{11.4:2}}

\noindent\apa{11.4 Example 3}\\
\apaex{\parencite{11.4:3}}\\
\apaex{\textcite{11.4:3}}

\noindent\apa{11.4 Example 4}\\
\apaex{\parencite{11.4:4}}\\
\apaex{\textcite{11.4:4}}

\noindent\apa{11.4 Example 5}\\
\apaex{\parencite{11.4:5}}\\
\apaex{\textcite{11.4:5}}

\noindent\apa{11.4 Example 6}\\
\apaex{\parencite{11.4:6}}\\
\apaex{\textcite{11.4:6}}

\noindent\apa{11.4 Example 7}\\
\apaex{\parencite{11.4:7}}\\
\apaex{\textcite{11.4:7}}

\printbibliography[title={APA 11.4 References}]
\end{refsection}
\newpage
\subsection{APA 11.5--Statues (Laws and Acts)}

\begin{refsection}

\noindent\apa{11.5 Example 8}\\
\apaex{\parencite{11.5:8}}\\
\apaex{\textcite{11.5:8}}

\noindent\apa{11.5 Example 9}\\
\apaex{\parencite{11.5:9}}\\
\apaex{\textcite{11.5:9}}

\noindent\apa{11.5 Example 10}\\
\apaex{\parencite{11.5:10}}\\
\apaex{\textcite{11.5:10}}

\noindent\apa{11.5 Example 11}\\
\apaex{\parencite{11.5:11}}\\
\apaex{\textcite{11.5:11}}

\noindent\apa{11.5 Example 12}\\
\apaex{\parencite{11.5:12}}\\
\apaex{\textcite{11.5:12}}

\noindent\apa{11.5 Example 13}\\
\apaex{\parencite{11.5:13}}\\
\apaex{\textcite{11.5:13}}

\printbibliography[title={APA 11.5 References}]
\end{refsection}
\newpage
\subsection{APA 11.6--Legislative Materials}

\begin{refsection}

\noindent\apa{11.6 Example 14}\\
\apaex{\parencite{11.6:14}}\\
\apaex{\textcite{11.6:14}}

\noindent\apa{11.6 Example 15}\\
\apaex{\parencite{11.6:15}}\\
\apaex{\textcite{11.6:15}}

\noindent\apa{11.6 Example 16}\\
\apaex{\parencite{11.6:16}}\\
\apaex{\textcite{11.6:16}}

\noindent\apa{11.6 Example 17}\\
\apaex{\parencite{11.6:17}}\\
\apaex{\textcite{11.6:17}}

\noindent\apa{11.6 Example 18}\\
\apaex{\parencite{11.6:18}}\\
\apaex{\textcite{11.6:18}}

\printbibliography[title={APA 11.6 References}]
\end{refsection}
\newpage
\subsection{APA 11.7--Administrative and Executive Materials}

\begin{refsection}

\noindent\apa{11.7 Example 19}\\
\apaex{\parencite{11.7:19}}\\
\apaex{\textcite{11.7:19}}

\noindent\apa{11.7 Example 20}\\
\apaex{\parencite{11.7:20}}\\
\apaex{\textcite{11.7:20}}

\noindent\apa{11.7 Example 21}\\
\apaex{\parencite{11.7:21}}\\
\apaex{\textcite{11.7:21}}


\printbibliography[title={APA 11.7 References}]
\end{refsection}
\newpage
\subsection{APA 11.8--Patents}

\begin{refsection}

\noindent\apa{11.8 Example 22}\\
\apaex{\parencite{11.8:22}}\\
\apaex{\textcite{11.8:22}}

\printbibliography[title={APA 11.8 References}]
\end{refsection}
\newpage
\subsection{APA 11.9--Constitutions and Charters}

\begin{refsection}

\noindent\apa{11.9 Example 23}\\
\apaex{\parencite{11.9:23}}\\
\apaex{\textcite{11.9:23}}

\noindent\apa{11.9 Example 24}\\
\apaex{\parencite{11.9:24}}\\
\apaex{\textcite{11.9:24}}

\noindent\apa{11.9 Example 25}\\
\apaex{\parencite{11.9:25}}\\
\apaex{\textcite{11.9:25}}

\noindent\apa{11.9 Example 26}\\
\apaex{\parencite{11.9:26}}\\
\apaex{\textcite{11.9:26}}

\noindent\apa{11.9 Example 27}\\
\apaex{\parencite{11.9:27}}\\
\apaex{\textcite{11.9:27}}

\noindent\apa{11.9 Example 28}\\
\apaex{\parencite{11.9:28}}\\
\apaex{\textcite{11.9:28}}

\printbibliography[title={APA 11.9 References}]
\end{refsection}
\newpage
\subsection{APA 11.10--Treaties and International Conventions}

\begin{refsection}

\noindent\apa{11.10 Example 29}\\
\apaex{\parencite{11.10:29}}\\
\apaex{\textcite{11.10:29}}

\printbibliography[title={APA 11.9 References}]
\end{refsection}

\end{document}


% Local Variables:
% TeX-engine: luatex
% End:
