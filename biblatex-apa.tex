% $Id$
\documentclass{ltxdockit}
\usepackage[utf8]{inputenc}
\usepackage[british]{babel}
\usepackage[strict,babel=once]{csquotes}
\usepackage{shortvrb}
\usepackage{ifthen}
\MakeAutoQuote{«}{»}
\MakeShortVerb{\|}

\newcommand\apa[2][]{\ifthenelse{\equal{#1}{}}%
                       {\texttt{(APA #2)}}%
                       {\texttt{(APA #2 Example #1)}}}


\titlepage{%
  title={APA \sty{biblatex} style},
  subtitle={Citation and References macros for \sty{biblatex}},
  url={http://www.ctan.org/tex-archive/macros/latex/exptl/biblatex-contrib/biblatex-apa/},
  author={Philip Kime},
  email={Philip@kime.org.uk},
  revision={0.6},
  date={\today}}

\hypersetup{%
  pdftitle={APA \sty{biblatex} style},
  pdfsubject={APA Citation and References macros for \sty{biblatex}},
  pdfauthor={Philip Kime},
  pdfkeywords={latex, biblatex, APA, style}}

\begin{document}

\printtitlepage
\tableofcontents

\section{Introduction}
\label{int}

\subsection{About}

This package is a \sty{biblatex} style for APA (American Psychological
Association) style compliant documents typeset in \latex. It implements a
citation style (\path{apa.cbx}), a references section style
(\path{apa.bbx}) and string localisation files (\path{<language>-apa.lbx}).
Currently there are only string localisations for a few languages---if you can help with any other languages, please mail me; the
localisation |.lbx| files are very small and simple and it would be a small
translation task for the few APA-specific strings needed.
The styles are loaded just like any other \sty{biblatex} styles but I
wouldn't try to use the citation and references styles separately as they
rely on each other, macro-wise, in places.

In this document and in the code, the specific APA requirements are
referred to by the section and (if appropriate) the example number of the
APA Style Guide (5th Edition).

\subsection{Requirements}\label{ref:req}

You will need to be using \sty{csquotes} ($\geq$ 4.3) and \sty{biblatex}
($\geq$ 0.8b). If you want to take advantage of the \sty{biblatex}
|\DeclareQuotePunctuation| facility to enforce the APA required «American»
punctuation, you should normally use the \sty{babel} package with the «american»
option (see \sty{biblatex} manual section |3.9.1|). You can of course use
other languages but in such cases, to adhere to APA «American» punctuation
rules (following commas moved inside closing quotes etc.), then you should
set up |\DeclareQuotePunctuation| yourself as per section |4.6.5| of the
\sty{biblatex} manual.

If you are using the |apa.cls| \latex class, you need be using version
$\geq$ 1.3.4. The class should be invoked with the |noapacite| class option
as per the |apa.cls| documentation. Without this class option, the
|apa.cls| class will automatically try to use plain \bibtex with the
|apacite| style which is completely incompatible with \sty{biblatex}.

\subsection{License}

Permission is granted to copy, distribute and/or modify this software under
the terms of the \latex Project Public License, version
1.3c\footnote{\url{http://www.latex-project.org/lppl.txt}}. The current
maintainer is Philip Kime (\textcopyright\ 2009).

\subsection{History}

When I started using \sty{biblatex}, I assumed there would be an APA style
when I went looking. I was wrong. I started to try to make one and realised
why there was none. The APA style manual is enormous; the citation and
references specifications run to about 60 pages and are very specific in
terms of formatting. They are also not entirely consistent but then again,
it is rare to have such a thorough specification to work from. There are
some parts of the requirements which are impossible to automatically
satisfy given the current reliance of \sty{biblatex} (as of version
\texttt{0.8}) on \bibtex . These limitations are described below.

\subsection{Acknowledgments}

I wished I could acknowledge someone as then this wouldn't have been such a
huge piece of work \ldots\ thanks to Philipp Lehman for \sty{biblatex} which
really has been a major advance over pure \bibtex.

\section{Use}\label{ref:use}
\label{use}
Put the \path{.cbx}, \path{.bbx} and \path{.lbx} files in your texmf tree, usually:\\

\noindent\path{<texmf>/tex/latex/biblatex/cbx/apa.cbx}\\
\path{<texmf>/tex/latex/biblatex/bbx/apa.bbx}\\
\path{<texmf>/tex/latex/biblatex/lbx/<language>-apa.lbx}\\

\noindent Specify the style in the usual way when loading \sty{biblatex}. 

\begin{ltxcode}
\usepackage[american]{babel}
\usepackage{csquotes}
\usepackage[style=apa]{biblatex}
\DeclareLanguageMapping{american}{american-apa}
\end{ltxcode}

\noindent Note that the APA manual requires the forcing of titles into
«sentence case», that is, initial cap followed by lower case for sentence
units, with the exception of names and material from languages which do not
follow English capitalisation. As of version 0.8a, \sty{biblatex} has a
|\MakeSentenceCase| macro which deals with this. So, in the traditional
\bibtex way, capitalise correctly in the |.bib| file, protecting names etc.
with the usual brace pairs and the style will take care of forcing the
APA-style sentence case in the References section. Unlike the References
section, titles in citations in the APA style appear in normal case and the
style will ensure this too.

\subsection{Localisation}

Localisation is provided for APA-specific strings in the accompanying
|.lbx| files. To use these files, put an appropriate mapping in your
document preamble, after loading \sty{biblatex}.

Normal use will invoke babel with the «american» babel language. After
loading babel and biblatex, put this in the preamble (see full preamble
example in section \ref{ref:use} above):

\begin{ltxcode}
\DeclareLanguageMapping{american}{american-apa}
\end{ltxcode}

The APA manual does not mention nor sanction any non «American» English
strings but it is quite common for non-English journals to ask for APA style
bibliographies and so this must be supported.

Here is an example to load the German localisation strings. This assumes
that you are using the \sty{babel} package with the appropriate language
option:

\begin{ltxcode}
\DeclareLanguageMapping{german}{german-apa}
\end{ltxcode}

This loads the |german-apa.lbx| file which in turn, loads the |german.lbx|
file and augments it with APA-specific strings. If you are not using the
babel «american» option, they you may need to set up the
|\DeclareQuotePunctuation| option as mentioned in section \ref{ref:req} above.
You would only need to do this if, for some strange reason, a non-american
journal required the (rather horrible) american practice of moving final
punctuation marks inside closing quotes.

\subsubsection{Date formats}

Obviously, an American style uses month/day/year formats. European formats
are usually day/month/year. The APA style dictates long date formats and so
this is not really apparent in the bibliography. There is an APA standard
for long date formats which may not be correct for European journals using
roughly APA formatting standards (German journals tend to have a different
long date format, for example). The «american» babel option will give you
APA compliant US date formats. The \sty{.lbx} files default to the US
format but contain an alternative |\DefineBibliographyExtras| command which
is more tuned to the likely required language date format. Simply comment the
default and uncomment the alternative block if you need this. There is a
comment in the file to mark the alternative date format code. Note that
such alternative date formats are not strictly APA compliant.

\subsection{Options}
\label{use:options}

\subsubsection{noremoteinfo}

This option can be set at a global level:

\begin{ltxcode}
\usepackage[noremoteinfo,style=apa]{biblatex}
\end{ltxcode}

or on a per-entry basis in the \path{.bib} file:

\begin{ltxcode}
OPTIONS = {noremoteinfo}
\end{ltxcode}

It has the effect of suppressing the printing of any URL and retrieval
information. This is because many people get their \path{.bib} entries from
services like PubMED which always include URL or DOI information in the
entries. Users may not want to always have this information displayed in
their References section since it clutters things up. By default,
\sty{biblatex-apa} will always display such information if it is available
in the \path{.bib} entry

\subsection{Limitations}
\label{use:limit}
There are certain limitations you need to bear in mind when using these
styles. The APA manual is written without any regard for automation of the
citation and references styles---it just tells you how it wants things to
look and the implicit assumption is that you would type out everything by
hand if necessary. Having said that, the vast majority of the APA citations
and references style is implemented, there are just a few exceptions which
are either hardly worth the coding pain for such rare cases or which are
impossible due to \sty{biblatex} limitations. It is likely that with future
\sty{biblatex} versions some or indeed all of these limitations may be
lifted but that's not likely until after \sty{biblatex} version
\texttt{1.0}, according to the |biblatex| author Philipp Lehman.

\subsubsection{Citation Limitations}

\begin{itemize}
\item \apa{3.95} Disambiguation of truncated author name lists. This is
  currently impossible to automate via a \sty{biblatex} style. The
  \sty{maxnames} and \sty{minnames} options of \sty{biblatex} only serve to
  specify the truncation limits but do not disambiguate name lists which
  truncate to the same string. This cannot be reliably done in the style
  since it would need to build a backwards and forward list of identical
  truncations and map these onto the internal \sty{fullhash} field of
  \sty{biblatex} in order to determine which truncations need
  disambiguating. Since there is no way of knowing, when you truncate
  something, whether something later in the document would have the same
  truncation (without \path{.aux} file support anyway), this is not really
  possible. It's also further complicated by the fact that \bibtex doesn't
  really support name lists in a way which would make this possible and
  since the main name list processing is done in \sty{biblatex} by \bibtex
  still, there isn't much that can be done about this at the moment until
  \sty{biblatex} moves to using something other than \bibtex for its data.
  See the following Usenet thread:\\

  \url{http://groups.google.com/group/comp.text.tex/browse_thread/thread/f3c50d7065159ad4#}\\

  A related issue is that the \sty{uniquename} option doesn't work in lists
  of authors, again because all the name processing is done by \bibtex.
  Currently, \sty{biblatex} disambiguates non-unique names in lists by
  using the \sty{labelyear} mechanisms which is against APA style (and most
  styles since this is supposed to disambiguate same author(s)/different
  year situations) but there's nothing really that can be done about it at
  the moment. Thankfully, these two problems are not that common and only
  occur when you have long lists of authors which share many members with
  variations in initials etc.
\item There are some anomalies like \apa[17]{4.16} where two years are
  supposed to be cited but one of the years is embedded in the \path{.bib}
  entry in information that is not parsed out. There isn't much to be done
  about this at the moment apart from copying some of the information into
  other fields like |ORIGYEAR| as in \apa[40]{4.16}. There isn't even that
  option in \apa[17]{4.16} because |ORIGYEAR| and |YEAR| are the opposite
  way round to all other such examples.
\item If you use |\marginpar| or other commands that reset the paragraph
  counter, then the code which tracks citations within paragraphs in order
  to decide whether to print a year or not will probably break. There is
  probably some esoteric way of making this not happen but I'm not enough
  of a \tex hacker to deal with this at the moment. Then again, the APA
  guide explicitly states that year elision is reset per paragraph and so
  if you add extra pseudo-paragraphs etc. it's probably your own fault if
  it breaks \ldots 
\end{itemize}

\subsubsection{Reference Section Limitations}

\begin{description}
\item\apa{4.08} Can't deal yet with authors listed as «with».
\item\apa[36]{4.16} Can't specify separate Volume and Series editors in
  this way. See the example in \path{biblatex-apa-test-references.bib} for
  an alternative that should be acceptable for APA style.
\item\apa[17,40]{4.16} It's rather hard and/or messy to automatically and
  properly format things like the additional «retrieved/reprint from»
  sections. It is possible using the |XREF| field and custom citation
  drivers in the references but this means setting \sty{mincrossrefs} to 1,
  which breaks the APA requirement to only include main-body cited works
  \apa{4.01}. It's also not obvious which field to use to contain the
  introductory string. Also, there is no way to automatically |XREF| the
  page range in the reprint specification in, for example, \apa[40]{4.16}.
  This will have to wait until \sty{biblatex} gets a real data model
  backend. The |ADDENDUM| field will have to do for now.
\end{description}

\section{Details}

The detailed information for this style is contained in the example document and
accompanying \path{.bib} files:
\begin{description}
\item[\path{biblatex-apa.tex}] This document.
\item[\path{biblatex-apa-test.tex}]\footnote{\path{biblatex-apa-test.pdf}
    is also provided and is the typeset version of this \latex source
    file.} This document typesets just about every useful example from
  \apa{3.94}--\apa{3.103} and \apa{4.01}--\apa{4.16}. The examples in it
  aim to look as much like the APA manual examples as possible. All
  citation examples in the document are real examples using a \path{.bib}
  file. This document also typesets every Reference section example from
  \apa{4.16}.
\item[\path{biblatex-apa-test-citations.bib}] This contains the \path{.bib}
  entries for the citations examples. You won't find anything of interest
  in this file---it's just used to provide real data for the citation
  examples.
\item[\path{biblatex-apa-test-references.bib}] This contains the
  \path{.bib} entries for all of the examples in \apa{4.16}. This file is
  the main documentation for the |biblatex-apa| implementation of the APA
  References section style. To see how the style deals with a particular
  example from \apa{4.16}, look it up in here. Every example is marked with
  the APA example number and has explanatory notes.
\item[\path{biblatex-apa.cbx}] The |biblatex-apa| citations style. It is
  decently structured with comments but shouldn't need to be read for
  normal use.
\item[\path{biblatex-apa.bbx}] The |biblatex-apa| references style. It is
  decently structured with comments but shouldn't need to be read for
  normal use.
\end{description}

\subsection{Citations}

\apa{3.103} requires that there should be no parentheses around the year of
the citation when the citation itself occurs within parenthesis. This would
be really too much to completely automate as it is within the remit of a
citation style since it requires knowledge of the current typesetting
state. So, the new citation command

\begin{ltxsyntax}
\cmditem{nptextcite}[prenote][postnote]{key}<punctuation>
\end{ltxsyntax}

\noindent is provided for such situations. It is identical to |\textcite|
but does not put parentheses around the year and separates items with
commas. See examples using this command in \path{biblatex-apa-test.tex}.

The requirement in \apa{3.95} for years only in the first cite in a
paragraph is tricky but was possible to do with some rather deep \tex magic
which nobody should be under any illusions about: I found out how to do it
by reading the UK \tex FAQ.

\subsection{References}

The references style was based on the \sty{biblatex} default
|authoryear-comp| style but is so heavily modified, it's almost
unrecognisable. Some general notes:
\begin{itemize}
\setlength{\itemsep}{0pt}
\item Many things in brackets at the end of the reference entries in the
  examples in \apa{4.16} can be dealt with by simply including them in the
  |ADDENDUM| \path{.bib} field. This shouldn't be necessary too often but
  sometimes it's the best (only) way of dealing with an example.
\item There are occasions where there is no sensible \path{.bib} key to
  use. This applies to things like |MOVIE| and |VIDEO| entries mainly. The
  format of these requires that different roles (Director, Producer etc.)
  are separately specified for different names. This is not really possible
  for the usual |AUTHOR| or |EDITOR| fields (again, another limitation of
  \bibtex, see \ref{use:limit} above). In such cases, I have resorted to
  the \sty{biblatex} custom |NAME| and |NAMETYPE| fields which are not very
  portable but until \bibtex is replaced by something more flexible, there
  is no way round this without making things very messy.
\item APA style sometimes refers to the «series» of a multi-volume work
  \apa[36]{4.16}. This corresponds to the |MAINTITLE| field in the
  \path{.bib} and \emph{not} the |SERIES| field.
\item In the entries, I sometimes use |DATE| and sometimes use
  |YEAR/MONTH/DAY|. These can generally be used interchangeably, see
  \sty{biblatex} docs. The same applies to |URLDATE| and
  |URLYEAR/URLMONTH/URLDAY|.
\item |VOLUME|, |NUMBER| and |CHAPTER| are forced into arabic numerals if
  they are given as roman numerals, as required by \apa{4.03}.
\end{itemize}

\noindent Examples from \apa{4.16} in |biblatex-apa-test-references.bib|
which cannot be dealt with cleanly are marked with the line

\begin{verbatim}
%%%%%%%%%%%%%%%%%%%%%%%%%% NOT COMPLETE %%%%%%%%%%%%%%%%%%%%%%%%%%
\end{verbatim}

\noindent and a comment explaining the issue.

\section{Revision history}

\begin{changelog}

\begin{release}{0.1}{2008-12-01}
\item Initial release
\end{release}

\begin{release}{0.2}{2008-12-06}
\item Added |noremoteinfo| option (see section \ref{use:options}).
\item Fixed bbx bug with more than 7 authors still printing names after «et al». Was
  due to resetting maxnames to 999.
\item Removed the customised (hacked) |apa-biblatex.cls| class from the package as
  |apa.cls| version 1.3.4 is compatible with \sty{biblatex}.
\item Altered documentation about requiring the «american» babel option.
  This is not required if you set up |\DeclareQuotePunctuation| yourself.
\item Added minimum required version of \sty{csquotes}.
\item Minor doc tweaks.
\end{release}

\begin{release}{0.3}{2008-12-21}
\item Updated to use new fields (|EVENTTITLE|) and new options
  (|usetranslator|) from \sty{biblatex} 0.8b.
\end{release}

\begin{release}{0.4}{2009-07-24}
\item Fixed bug where multiple year ranges were not displayed properly.
\item Updated to remove pre-\sty{biblatex} 0.8e macros error.
\end{release}

\begin{release}{0.5}{2009-09-19}
\item Replaced literal string with localised form in url macro.
\item Moved localisation strings into \sty{.lbx} files.
\item |\DeclareLanguageMapping| is now needed in preamble.
\item Moved |\DeclareBibliographyExtras| into \sty{.lbx} files.
\item Some support for alternative localised date formats.
\item Fixed slant/italic font problem since the |\mkbibemph| macro had
  changed in \sty{biblatex} 0.8e.
\item Fixed |liststop| error which was preventing name lists with two
  entries from having the comma before the ampersand.
\end{release}

\begin{release}{0.6}{2009-11-20}
\item Corrected two bugs in |cite| macro which left a trailing space after
  multiple cites and actually cited the year twice for multiple cite
  commands in some circumstances.
\item Corrected bug where |\textcite| would leave a stray open bracket on
  the stack when year was suppressed withing a paragraph.
\end{release}

\end{changelog}
\end{document}
