\documentclass{article}
\usepackage[utf8]{inputenc}
\usepackage[T1]{fontenc}
\usepackage[american]{babel}
\usepackage{csquotes}
\usepackage{shortvrb}
\usepackage{ifthen}
\usepackage{color}
\usepackage{MnSymbol}
\MakeShortVerb{\|}

% Biblatex
\usepackage[citestyle=apa,%
            bibstyle=apa,%
            bibtex8=true,%
            bibencoding=inputenc]{biblatex}

\newcommand\apa[2][]{\ifthenelse{\equal{#1}{}}%
                       {\textcolor{blue}{\texttt{(APA #2)}}}%
                       {\textcolor{blue}{\texttt{(APA #2 Example #1)}}}}

\setlength{\parskip}{3ex}
\bibliography{biblatex-apa-test-citations,biblatex-apa-test-references}

\def\apaex#1{\hbox{\hspace{-4em}\texttt{\small \detokenize{#1}}}\\ $\rcurvearrowse$ \textbf{#1}}
\def\apaexs#1{\hbox{\texttt{\footnotesize \detokenize{#1}}} \textbf{\small #1}}

% This just makes it easier to find a specific (APA 4.16) example in the
% typeset references section
\reversemarginpar
\renewbibmacro*{begentry}{\marginpar{\footnotesize \textcolor{blue}{\thefield{entrykey}}}}

%%%%%%%%%%%%%%%%%%%%%%%%%%%% END PREAMBLE %%%%%%%%%%%%%%%%%%%%%%%%%
\DeclareLanguageMapping{american}{american-apa}

\begin{document}

\section*{|biblatex-apa| style examples}

This file typesets just about all useful examples from
\apa{6.11}--\apa{6.21} and \apa{7.01}--\apa{7.11}. The references section
includes every example from \apa{7.11}. Please refer to the
|biblatex-apa-test-references.bib| file for details on the references
entries. The |.bib| key for each entry in the References section is listed
for convenience in the left margin. The keys are not arbitrary and consist
of the APA section they are taken from (7.01--7.11), followed by a colon,
followed by the example number. This makes it easier to cross-reference
the typeset examples with the commented |.bib| file. I chose not to put the
examples in the References section in example number order so that the APA
requirements for References list alphabetisation and order could also be
demonstrated.

\section*{Citations}

\noindent Please see accompanying file |biblatex-apa-test-citations.bib|
for the bibliographic entries which these examples use.

\noindent\apa{6.11}\\
Simple cite. ``Jr.'' suffix is not shown (bib entry for this example has a suffix):\\
\apaex{\textcite{6.11}}

\noindent Within a paragraph, not in the ``narrative sense'':\\
\apaex{\parencite{6.11}}

\noindent To cite the parts separately:\\
\apaex{\citeyear{6.11}, \citeauthor{6.11}}

\noindent The per-paragraph rules for elision of years are more flexible in
APA 6th edition. There is more discretion to do this as the narrative
consistency suggests and so this style no longer automatically elides years
aver mention after the first within a paragraph. Cases can be handled as
per the examples above.

\noindent\apa{6.12} \apa{6.13}\\
Citations like\\
\apaex{\textcite{6.12a}}\\
which have two authors are never name-truncated after the first cite:\\
\apaex{\textcite{6.12a}}

\noindent First citation of 3--5 author entry:\\
\apaex{\textcite{6.12b}}

\noindent Subsequent first citations in a paragraph:\\
\apaex{\textcite{6.12b}}\\
Subsequent citations within a paragraph:\\
\apaex{\citeauthor{6.12b}}

\noindent Note that the dropping of the year for subsequent paragraph
citations is not automatic as there may be cases where you don't want to do
this (see APA 6.11).

\noindent\textcolor{red}{Note: The name list disambiguation required in the
  \emph{Exception:} clause here in the APA manual cannot be automated
  currently in |biblatex|. This is due to the underlying reliance on the
  |bibtex| |.bib| data model. This will change in a future |biblatex|
  release. See |biblatex-apa| docs.}

\noindent Multiple-authors in running text are separated by ``and''.
However, in parenthetical cites, multiple authors are separated by ``\&'':\\
\apaex{\textcite{6.12e}}\\
\apaex{\parencite{6.12f}}

\noindent The following citation should be name truncated on first cite
since it has six or more authors:\\
\apaex{\textcite{6.12g}}

\noindent\textcolor{red}{Note: The note above applies to the disambiguation
of entries with six or more names}

\noindent Now, following the examples in Table 6.1, p. 177 of the APA
manual. Note in the code that typesets these examples, |\citereset| is used
to pretend that the parenthetical examples are the first in the text.

\begin{center}
\begin{tabular}{lllll}
\textbf{\parbox{2cm}{\center Type of citation}} & \textbf{\parbox{2.4cm}{\center First
    citation in text}} & \textbf{\parbox{2.4cm}{\center Subsequent citations in
    text}} & \textbf{\parbox{2.4cm}{\center Parenthetical format, first citation
    in text}} & \textbf{\parbox{2.4cm}{\center Parenthetical format, subsequent
    citations in text}}\\\\
\hline
\\
\parbox{2cm}{One work by one author}
& \parbox{2.4cm}{\apaexs{\textcite{6.13a}}} &
\parbox{2.4cm}{\apaexs{\textcite{6.13a}}}\citereset
& \parbox{2.4cm}{\apaexs{\parencite{6.13a}}}
& \parbox{2.4cm}{\apaexs{\parencite{6.13a}}}\\\\
\parbox{2cm}{One work by two authors}
& \parbox{2.4cm}{\apaexs{\textcite{6.13b}}} & 
\parbox{2.4cm}{\apaexs{\textcite{6.13b}}}\citereset
& \parbox{2.4cm}{\apaexs{\parencite{6.13b}}}
& \parbox{2.4cm}{\apaexs{\parencite{6.13b}}}\\\\
\parbox{2cm}{One work by three authors}
& \parbox{2.4cm}{\apaexs{\textcite{6.13c}}} & 
\parbox{2.4cm}{\apaexs{\textcite{6.13c}}}\citereset
& \parbox{2.4cm}{\apaexs{\parencite{6.13c}}}
& \parbox{2.4cm}{\apaexs{\parencite{6.13c}}}\\\\
\parbox{2cm}{One work by four authors}
& \parbox{2.4cm}{\apaexs{\textcite{6.13d}}} & 
\parbox{2.4cm}{\apaexs{\textcite{6.13d}}}\citereset
& \parbox{2.4cm}{\apaexs{\parencite{6.13d}}}
& \parbox{2.4cm}{\apaexs{\parencite{6.13d}}}\\\\
\parbox{2cm}{One work by five authors}
& \parbox{2.4cm}{\apaexs{\textcite{6.13e}}} & 
\parbox{2.4cm}{\apaexs{\textcite{6.13e}}}\citereset
& \parbox{2.4cm}{\apaexs{\parencite{6.13e}}}
& \parbox{2.4cm}{\apaexs{\parencite{6.13e}}}\\\\
\parbox{2cm}{One work by six authors}
& \parbox{2.4cm}{\apaexs{\textcite{6.13f}}} & 
\parbox{2.4cm}{\apaexs{\textcite{6.13f}}}\citereset
& \parbox{2.4cm}{\apaexs{\parencite{6.13f}}}
& \parbox{2.4cm}{\apaexs{\parencite{6.13f}}}\\\\
\parbox{2cm}{Groups (readily identified through abbreviation) as authors}
& \parbox{2.4cm}{\apaexs{\textcite{6.13g}}} & 
\parbox{2.4cm}{\apaexs{\textcite{6.13g}}}\citereset
& \parbox{2.4cm}{\apaexs{\parencite{6.13g}}}
& \parbox{2.4cm}{\apaexs{\parencite{6.13g}}}\\\\
\parbox{2cm}{Groups (no abbreviation) as authors}
& \parbox{2.4cm}{\apaexs{\textcite{6.13h}}} & 
\parbox{2.4cm}{\apaexs{\textcite{6.13h}}}\citereset
& \parbox{2.4cm}{\apaexs{\parencite{6.13h}}}
& \parbox{2.4cm}{\apaexs{\parencite{6.13h}}}\\\\
\hline
\end{tabular}
\end{center}

% \noindent\apa{3.96}\\
% First cite of an author with a |SHORTAUTHOR| field in the
% entry (usually an institution), shows the |AUTHOR| and also the |SHORTAUTHOR| field:\\
% \apaex{\textcite{3.96}}\\
% Thereafter, just the |SHORTAUTHOR|:\\
% \apaex{\textcite{3.96}}

% \noindent\apa{3.97}\\
% Use |SHORTTITLE| field of the entry if it exists:\\
% \apaex{\parencite{3.97a}}\\
% Books, reports etc. use italics instead of quotes:\\
% \apaex{\textcite{3.97b}}

% \noindent\apa{3.98}\\
% Citations of an entry with an author who shares a
% surname with another entry always appears with initials:\\
% \apaex{\textcite{3.98a}}\\
% and\\
% \apaex{\textcite{3.98b}}\\
% \textcolor{red}{Note: The second example in \apa{3.98} is currently
%   impossible to automate because of name list parsing in BibTeX. See
%  |biblatex-apa| docs}.

% \noindent\apa{3.99}\\
% Citations of different works by the same author in a
% compact style:\\
% \apaex{\parencite{3.99a,3.99b}}\\
% \apaex{\parencite{3.99c,3.99d,3.99e}}

% \noindent Citations of works by same authors in the same
% year:\\
% \apaex{\parencite{3.99f,3.99g,3.99h,3.99i,3.99j,3.99k}}

% \noindent Compact citations in alphabetic order:\\
% \apaex{\parencite{3.99l,3.99m,3.99n}}

% \noindent Compact citations with special order:\\
% \apaex{\parencites{3.99o}[see also][]{3.99p,3.99q}}

% \noindent\apa{3.100}\\
% Old works with no sensible year use translation
% postnote.\\
% \textcolor{red}{Note: The example in the APA manual cannot be
%   automatically format as presented as it is using a citation infix
%   (``trans.'' between the name and year) which is not supported by
%   |biblatex|. However, it also allows a postnote after the year, which is fine}:\\
% \apaex{\parencite[][Pakaluk translation]{3.100a}}\\
% Entries with an |ORIGYEAR| field will automatically use it:\\
% \apaex{\textcite{3.100b}}

% \noindent\apa{3.101}\\
% These examples are easily dealt with using standard |biblatex| functionality.\\
% \apaex{\parencite[][332]{3.101a}}\\
% \apaex{\parencite[][chap. 3]{3.101b}}\\
% \apaex{\parencite[][5]{3.101c}}\\
% \apaex{\parencite[][Conclusion section, para. 1]{3.101d}}

% \noindent\apa{3.103}\\
% Within parentheses, use the |\nptextcite| command which is equivalent to
% the |\textcite| command but omits the parenthesis and uses commas instead.
% See the |biblatex-apa| docs. The example from the APA manual:\\
% \apaex{(\nptextcite[see Table 2 of][]{3.103a} for complete data)}

% \subsection*{Citation examples mentioned in the APA guidelines on References}

% \noindent Please see accompanying file |biblatex-apa-test-references.bib|
% for the bibliographic entries which these examples use.

% \noindent\apa[4]{4.16}\\
% Citation of a work with more than six authors:\\
% \apaex{\parencite{4.16:4}}

% \noindent\apa[5]{4.16}\\
% Citation of a work in press:\\
% \apaex{\parencite{4.16:5}}

% \noindent\apa[8]{4.16}\\
% Citation of a work with no author/editor:\\
% \apaex{\parencite{4.16:8}}

% \noindent\apa[9]{4.16}\\
% Citation of a work with no author/editor and truncated title:\\
% \apaex{\parencite{4.16:9}}

% \noindent\apa[12]{4.16}\\
% This would be dealt with in the same way as examples 8 and 9, with a
% |SHORTTITLE| field.

% \noindent\apa[17]{4.16}\\
% \textcolor{red}{Note: The citation of both years cannot be done
%   automatically as such entries don't have their |NOTE| field parsed in
%   order to extract the retrieval year. Also, we can't really use |ORIGYEAR|
%   in such entries as the position of the |ORIGYEAR| and |YEAR| information
%   is the opposite way round to all other such examples (examples 32 and 39)}.

% \noindent\apa[22]{4.16}\\
% Secondary sources can be dealt with using standard |biblatex|
% functionality:\\
% \apaex{(\nptextcite[as cited in][]{4.16:22})}

% \noindent\apa[26]{4.16}\\
% Titles of books with no author/editor again. Note that in the APA manual,
% this example is incorrect as the citation is not italics as required by \apa{3.97}:\\
% \apaex{\parencite{4.16:26}}

% \noindent\apa[28]{4.16}\\
% Year ranges in entry--nothing special here, just uses the |YEAR| field:\\
% \apaex{\parencite{4.16:28}}

% \noindent\apa[29]{4.16}\\
% Note that the style defines some macros you can use to format the short
% forms of the DSM version titles. See |biblatex-apa-test-references.bib|.
% First mention of a DSM edition:\\
% \apaex{\textcite{4.16:29}}\\
% Subsequently, the |SHORTHAND| field is used:\\
% \apaex{\textcite{4.16:29}}

% \noindent\apa[32]{4.16}\\
% Translation with |ORIGYEAR| field in entry:\\
% \apaex{\parencite{4.16:32}}

% \noindent\apa[35]{4.16}\\
% Citation for work in press:\\
% \apaex{\parencite{4.16:35}}

% \noindent\apa[39]{4.16}\\
% Another |ORIGYEAR| example:\\
% \apaex{\parencite{4.16:39}}

% \noindent\apa[40]{4.16}\\
% A rather clumsy |ORIGYEAR| example since we simply duplicate the |ORIGYEAR|
% information from the |ADDENDUM| field which is not parsed. There is no
% other way to do this at present. It's suspicious also because the
% |ORIGYEAR| field doesn't really contain the original year but the year of
% the reprint. However, it's not so bad since the example citation requires
% the year of the reprint and the year of the cited work and that's what we have:\\
% \apaex{\parencite{4.16:40}}

% \noindent\apa[69]{4.16}\\
% Citing music:\\
% \apaex{\citetitle{4.16:69b} \parencite[][track 5]{4.16:69b}}

\newpage
\nocite{*}
% Exclude the citation examples from the References section - only want to
% see (APA 4.16) examples there.
\printbibliography[notkeyword=noinclude]
\end{document}















